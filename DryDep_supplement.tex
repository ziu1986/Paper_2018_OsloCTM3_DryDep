\documentclass[manuscript]{copernicus}
\graphicspath{{pictures/}}          % graphics
\begin{document}
\clearpage
\setcounter{page}{1}

\section*{Supplement}
\subsection*{S.1 Dynamic viscosity of air and quasi-laminar resistance}
\subsubsection*{S.1.1 Sutherland's law and linear fits}
\appendixfigures
\begin{center}
  \includegraphics[height=0.4\textheight]{pictures/dynamic_viscosity.pdf}
\end{center}
\subsubsection*{S.1.2 Comparison of $z_o$ and $R_b$ for $\mu(T) = m\cdot T$ and Sutherland's law: \chem{H_2O}}
\appendixfigures
\begin{center}
  \includegraphics[height=0.4\textheight]{pictures/dynamic_viscosity_zo_Rb.pdf}
\end{center}
\subsubsection*{S.1.3 $R_b$ for different species}
\appendixfigures
\begin{center}
  \includegraphics[height=0.45\textheight]{pictures/dynamic_viscosity_Rbi_1.pdf}
  \includegraphics[height=0.45\textheight]{pictures/dynamic_viscosity_Rbi_2.pdf}
\end{center}

%\begin{minipage}[t]{1.\textwidth}
\subsection*{S.2 Tabulated parameters for stomatal conductance}
\appendixtables
\begin{sidewaystable*}[!htbp]
  %\caption{S.2 Tabulated parameters for stomatal conductance}
  \begin{tabular}{lcccccccccccccc}
    \tophline
    Code$^*$ & $g_\text{max}$ & $f_\text{min}$ & $\phi_a$ & $\phi_b$ & $\phi_c$ & $\phi_d$ & $\phi_e$ & $\phi_f$ & $\phi_\text{AS}$ & $\phi_\text{AE}$ & $\alpha_\text{light}$ & $T_\text{min}$ & $T_\text{opt}$ & $T_\text{max}$ \\
    & (\unit{mmol\,s^{-1}\,m^{-2}}) &&&&&&(\unit{days})&(\unit{days})&(\unit{days})&(\unit{days})&&(\unit{^\circ C})&(\unit{^\circ C})&(\unit{^\circ C})\\
    \middlehline
    CF & 140 & 0.1 & 0.8 & 0.8 & 0.8 & 0.8 & 1 & 1 & 0 & 0 & 0.006 & 0 & 18 & 36  \\
    DF & 150 & 0.1 & 0 & 0 & 1 & 0 & 20 & 30 & 0 & 0 & 0.006 & 0 & 20 & 35  \\
    NF & 200 & 0.1 & 1 & 1 & 0.2 & 1 & 130 & 60 & 80 & 35 & 0.013 & 8 & 25 & 38 \\
    BF & 200 & 0.02 & 1 & 1 & 0.3 & 1 & 130 & 60 & 80 & 35 & 0.009 & 1 & 23 & 39 \\
    TC & 300 & 0.1 & 0.1 & 1 & 0.1 & 0 & 45 & 0 & 0 & 0.0105 & 0.01 & 12 & 26 & 40 \\
    MC & 300 & 0.019 & 0.1 & 0.1 & 1 & 0.1 & 0 & 45 & 0 & 0 & 0.0048 & 0 & 25 & 51  \\
    RC & 360 & 0.02 & 0.2 & 0.2 & 1 & 0.2 & 20 & 45 & 0 & 0 & 0.0023 & 8 & 24 & 50  \\
    SNL & 60 & 0.01 & 1 & 1 & 1 & 1 & 1 & 1 & 0 & 0 & 0.009 & 1 & 18 & 36 \\
    GR & 270 & 0.01 & 1 & 1 & 1 & 1 & 0 & 0 & 0 & 0 & 0.009 & 12 & 26 & 40 \\
    MS & 200 & 0.01 & 1 & 1 & 0.2 & 1 & 130 & 60 & 80 & 35 & 0.012 & 4 & 20 & 37 \\
    WE & 0 & 1 & 1 & 1 & 1 & 1 & 1 & 1 & 1 & 1 & 1 & 0 & 1 & 0 \\
    TU & 0 & 1 & 1 & 1 & 1 & 1 & 1 & 1 & 1 & 1 & 1 & 0 & 1 & 0  \\
    DE & 0 & 1 & 1 & 1 & 1 & 1 & 1 & 1 & 1 & 1 & 1 & 0 & 1 & 0  \\
    W & 0 & 1 & 1 & 1 & 1 & 1 & 1 & 1 & 1 & 1 & 1 & 0 & 1 & 0 \\
    ICE & 0 & 1 & 1 & 1 & 1 & 1 & 1 & 1 & 1 & 1 & 1 & 0 & 1 & 0 \\
    U & 0 & 1 & 1 & 1 & 1 & 1 & 1 & 1 & 1 & 1 & 1 & 0 & 1 & 0 \\
    \bottomhline
  \end{tabular}
  \\
  \vspace{2\baselineskip}
  \begin{tabular}{lcccccccccccccc}
    \tophline
    Code$^*$ & $D_\text{max}$ & $D_\text{min}$ & $D_\text{crit}$ & $R_\chem{SO}$ & $R_\chem{O_3}$ & $h$ & $d_\text{SGS}$ & $d_\text{EGS}$ & $\nabla d_\text{SGS}$ & $\nabla d_\text{EGS}$ & $\text{LAI}_\text{min}$ & $\text{LAI}_\text{max}$ & LS & LE \\
    &(\unit{kPA})&(\unit{kPa})&(\unit{kPa})&(\unit{s\,m^{-1}})&(\unit{s\,m^{-1}})&(\unit{m})&(\unit{day})&(\unit{day})&(\unit{days/^\circ lat})&(\unit{days/^\circ lat})&(\unit{m^2\,m^{-2}})&(\unit{m^2\,m^{-2}})&(\unit{days})&(\unit{days}) \\
    \middlehline
    CF & 0.5 & 3 & 1000 & 0 & 200 & 20 & 0 & 366 & 0 & 0 & 5 & 5 & 1 & 1\\
    DF & 1 & 3.25 & 1000 & 0 & 200 & 20 & 100 & 307 & 1.5 & -2.0 & 0 & 4 & 20 & 30\\
    NF & 1 & 3.2 & 1000 & 0 & 200 & 8 & 0 & 366 & 0 & 0 & 4 & 4 & 1 & 1\\
    BF & 2.2 & 4 & 1000 & 0 & 200 & 15 & 0 & 366 & 0 & 0 & 4 & 4 & 1 & 1\\
    TC & 1.2 & 3.2 & 8 & 0 & 200 & 1 & 123 & 213 & 2.57 & 2.57 & 0 & 3.5 & 70 & 20\\
    MC & 1 & 2.5 & 1000 & 0 & 200 & 2 & 123 & 237 & 2.57 & 2.57 & 0 & 3 & 70 & 44\\
    RC & 0.31 & 2.7 & 10 & 0 & 200 & 1 & 130 & 250 & 0 & 0 & 0 & 4.2 & 35 & 65\\
    SNL & 1.3 & 3 & 1000 & 0 & 400 & 0.5 & 0 & 366 & 0 & 0 & 2 & 3 & 192 & 96\\
    GR & 1.3 & 3 & 1000 & 0 & 1000 & 0.3 & 0 & 366 & 0 & 0 & 2 & 3.5 & 140 & 135\\
    MS & 1.3 & 3.2 & 1000 & 0 & 200 & 2 & 0 & 366 & 0 & 0 & 2.5 & 2.5 & 1 & 1\\
    WE  & 1 & 0 & 1000 & 50 & 400 & 0.5 & 0 & 366 & 0 & 0 & 0 & 0 & 0 & 0\\
    TU & 1 & 0 & 1000 & 500 & 400 & 0.5 & 0 & 366 & 0 & 0 & 0 & 0 & 0 & 0\\
    DE & 1 & 0 & 1000 & 1000 & 2000 & 0 & 0 & 366 & 0 & 0 & 0 & 0 & 0 & 0\\
    W  & 1 & 0 & 1000 & 1 & 2000 & 0 & 0 & 366 & 0 & 0 & 0 & 0 & 0 & 0\\
    ICE & 1 & 0 & 1000 & 1000 & 2000 & 0 & 0 & 366 & 0 & 0 & 0 & 0 & 0 & 0\\
    U & 1 & 0 & 1000 & 400 & 400 & 10 & 0 & 366 & 0 & 0 & 0 & 0 & 0 & 0\\
    \bottomhline
  \end{tabular}
  \belowtable{$^*$: CF -- temperate/boreal coniferous; DF -- temperate/boreal deciduous; NF -- Mediterranean needleleaf; BF -- Mediterranean broadleaf; TC -- temperate crop; MC -- Mediterranean crop; RC -- root crop; SNL -- moorland; GR -- grass; MS -- Mediterranean scrub; WE -- wetlands; TU -- tundra; DE -- desert; W -- water; ICE -- ice.}
\end{sidewaystable*}
%\end{minipage}

\subsection*{S.3 Latitude dependent vegetation height}
\appendixfigures
\begin{center}
  \includegraphics[height=0.4\textheight]{pictures/tree_height.pdf}
\end{center}

\subsection*{S.4 Temperature dependent greening season}
\subsubsection*{S.4.1 Python code snippets}
\begin{verbatim}
#----------------------------------------------------------------------------------
import numpy as np
import pandas as pd
import xarray as xr
#----------------------------------------------------------------------------------
def start_growing_season_fixed(lat, **kwargs):
    '''
    Begin of growing season parameterization adapted from SMOKE-BEIS.
    '''
    bLeap = kwargs.pop('leap', False)
    if (bLeap):
       MAY31 = 152
       NOV1  = 306
       DEC31 = 366
    else:
       MAY31 = 151
       NOV1  = 305
       DEC31 = 365
    
    if lat < -65:
        return((0,))
    elif lat < -23:
        return((NOV1,1))
    elif lat <= 23:
        return((1,))
    elif lat < 65:
        return(((lat-23)*4.5,))
    else:
        return((0,))
#----------------------------------------------------------------------------------
def end_growing_season_fixed(lat, **kwargs):
    '''
    End of growing season parameterization adapted from SMOKE-BEIS.
    '''
    bLeap = kwargs.pop('leap', False)
    if (bLeap):
       MAY31 = 152
       NOV1  = 306
       DEC31 = 366
    else:
       MAY31 = 151
       NOV1  = 305
       DEC31 = 365
       
    if lat < -65:
        return((0,))
    elif lat < -23:
        return((DEC31,MAY31))
    elif lat <= 23:
        return((DEC31,))
    elif lat < 65:
        return((DEC31-(lat-23)*3.3,))
    else:
        return((0,))
#----------------------------------------------------------------------------------
def growing_season(temperature,**kwargs):
    '''
    Agricultural rule of thumb definition of growing season:
    5 consecutive days above 5 degree Celsius 
    and vice versa for end of growing season.
    '''
    # Shift start day of evaluation.
    s_shift = kwargs.pop('s_shift',365/2)
    # Number of days that need to fulfill temperature criteria
    degree_days_crit = kwargs.pop('ddc', 5)
    # Temperature criteria
    temperature_crit = kwargs.pop('tc', 5)
    # Switch to southern hemisphere evaluation
    sh = kwargs.pop('sh', False)
    # Activate verbose
    verbose = kwargs.pop('verbose', False)
    # Counters
    count_gdd = 0
    count_days = 0
    start_gs = (0,False)
    end_gs = (0,False)
    
    if sh:
    # Shift the temperature by halve a year - shifted back later.
        temp = temperature.roll(time=s_shift)
    else:
        temp = temperature
    for itemp in temp:
        count_days += 1
        if not start_gs[1]:
            if itemp > temperature_crit:
                count_gdd += 1
            else:
                count_gdd = 0
            if count_gdd == degree_days_crit:
                if sh:
                    start_gs = (count_days+s_shift, True)
                else:
                    start_gs = (count_days, True)
                count_gdd = 0
        elif start_gs[1] and not end_gs[1] and count_days>s_shift:
            if itemp <= temperature_crit:
                count_gdd += 1
                if verbose:
                    print(count_days, count_gdd)
            else:
                count_gdd = 0
            if count_gdd == degree_days_crit:
                if sh:
                    end_gs = (count_days-s_shift, True)
                else:
                    end_gs = (count_days, True)
                count_gdd = 0
       
    return({'sgs':start_gs,'egs':end_gs})
#----------------------------------------------------------------------------------
def growing_season_stadyn(xr_temp):
    '''
    Preprocess the greening season for OsloCTM3.
    Setting using the 5deg-5days criteria and in case this fails
    or is out of the defined bounds,
    fall back to SMOKE-BEIS parameterization.
    '''
    # Fetch leap year from input data
    bLeap = False
    if xr_temp.time.size==366:
       bLeap = True
    # Use 5deg-5days criteria   
    if (xr_temp.lat > 45 and xr_temp.lat < 85):
        gs = growing_season(xr_temp)
        sgs = (gs['sgs'][0],)
        egs = (gs['egs'][0],)
        if ( not gs['sgs'][1] or not gs['egs'][1] or sgs[0]>=egs[0]):
            # Check for failing 5deg-5days criteria => Fall back to fixed
            sgs = start_growing_season_fixed(xr_temp.lat.data, leap=bLeap)
            egs = end_growing_season_fixed(xr_temp.lat.data, leap=bLeap)
    elif (xr_temp.lat < -35 and xr_temp.lat >= -65):
        gs = growing_season(xr_temp, sh=True)
        sgs = (gs['sgs'][0],1)
        egs = (xr_temp.size, gs['egs'][0])
        # Check for failing 5deg-5days criteria => Fall back to fixed
        if (not gs['sgs'][1] or not gs['egs'][1] or sgs[0]>=egs[0]):
            sgs = start_growing_season_fixed(xr_temp.lat.data, leap=bLeap)
            egs = end_growing_season_fixed(xr_temp.lat.data, leap=bLeap)
    # Use SMOKE-BEIS parameterization
    else:
        sgs = start_growing_season_fixed(xr_temp.lat.data, leap=bLeap)
        egs = end_growing_season_fixed(xr_temp.lat.data, leap=bLeap)
    # Generate GDAY and GLEN fields
    # in case no growing season has been allocated
    if (sgs[0]==egs[0]==0):
        gday = np.repeat(0,len(xr_temp.time))
        glen = 0
    # in normal cases
    else:
        # handle northern hemisphere
        if len(sgs)==1:
            gday = np.concatenate((np.repeat(0,int(sgs[0])-1),
                                   np.arange(1,int(egs[0])-int(sgs[0])+1),
                                   np.repeat(0,len(xr_temp.time)-
                                   int(egs[0])+1)))
            glen = int(egs[0])-int(sgs[0])+1
        # handle southern hemisphere
        else:
            gday = np.concatenate((np.arange(1,int(egs[1])+1)+len(xr_temp.time)
                                   -int(sgs[0])+1,
                                   np.repeat(0,int(sgs[0])-int(egs[1])),
                                   np.arange(1,len(xr_temp.time)-int(sgs[0])+1)))
            glen = len(xr_temp.time)-(int(sgs[0])-int(egs[1]))+1
    # This should not happen, but in case it does, print info.
    if not(len(gday)==len(xr_temp.time)):
        print(xr_temp, sgs, egs)
    # Output
    data_gday = xr.DataArray(gday.astype(int),[('time', xr_temp.time.data)])
    return(data_gday, (int)(glen))
#----------------------------------------------------------------------------------
\end{verbatim}
%\subsubsection*{S.4.2 GLEN: NH mid--high latitudes only}
%\appendixfigures
%\begin{center}
%  \includegraphics[height=0.4\textheight, clip, trim={0.cm 1.75cm 0.cm 1.75cm}]{pictures/GLEN_2005.pdf}
%\end{center}

\subsection*{S.5 De-accumulation of photosynthetic active radiation (PAR) from OpenIFS: January 2nd 2005}
\subsubsection*{S.5.1 Output from OpenIFS: Accumulated PAR}
\appendixfigures
\begin{center}
  \includegraphics[height=0.4\textheight, clip, trim={0.7cm 0.7cm 0.7cm 0.7cm}]{pictures/PAR_as_is.png}
\end{center}
\subsubsection*{S.5.2 Partly de-accumulated}
\appendixfigures
\begin{center}
  \includegraphics[height=0.4\textheight, clip, trim={0.7cm 0.7cm 0.7cm 0.7cm}]{pictures/PAR_partly_deaccumulated.png}
\end{center}
\subsubsection*{S.5.3 De-accumulation of hour 00}
\appendixfigures
\begin{center}
  \includegraphics[height=0.4\textheight, clip, trim={0.75cm 0.75cm 0.75cm 0.75cm}]{pictures/PAR_deaccumulating_00h.png}
\end{center}

\subsection*{S.6 Average ozone dry deposition velocity seperated by land surface type (southern hemisphere)}
\appendixfigures
\begin{center}
  \includegraphics[height=0.4\textheight, clip, trim={0.7cm 0.7cm 0.7cm 0.55cm}]{pictures/final-total_ozone_drydepvelo_monthly_pft_sh}
\end{center}
\end{document}
