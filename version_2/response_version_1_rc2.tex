% Authors' response
\documentclass{scrartcl}
\usepackage[utf8]{inputenc}
\usepackage[english]{babel}
%\usepackage[T1]{fontenc}
\usepackage{amsmath}
\usepackage{xcolor}
\usepackage{url}
\usepackage{multirow}
\usepackage{graphicx}
\usepackage{hyperref}
\thispagestyle{empty}

\begin{document}
\section*{Authors' response}
To gmd-2019-21 referee\#2 David Simpson (10 Apr 2019):
We apologize for the confusion the misuse of the terms \emph{EMEP} and \emph{EMEP scheme} may have caused. We evaluated the major concerns raised in the review and found them severe enough that we have to revise our model and repeat the model experiments to address all concerns. Nevertheless, we will respond to all of the questions and concerns in more detail in the following.

\section{Major points}
\begin{itemize}
\item {\color{blue} Mosaic formulation:}
  \begin{itemize}
  \item {\color{blue} CTM3 claims to implement a mosaic approach, but instead of calculating deposition
rates over each land-cover, and then aggregating using Eqn.3, CTM3 seems to perform the following steps:
\begin{align}
  \label{eq:ra}
  R_a &= \frac{u_z}{u^2_*}\\
  \label{eq:Gsto}
  \overline{Gs^i_g(z)} &= \Sigma^N_{k=1} f_k \times Gs^i_{g,k}(z)\\
  \label{eq:Gns}
  \overline{Gns^i_g(z)} &= \Sigma^N_{k=1} f_k \times Gns^i_{g,k}(z)
\end{align}
As far as I can tell from the text, $R_a$ is calculated once, with the same value of $u_*$ for all land-covers. The CTM3 $R_b$ calculation seems to also use the same $u_*$ over different land-covers, except over sea where a more sophisticated scheme is used. Equations 2--3 above are equivalent to Eqns (22) and (26) from the manuscript. Now I am puzzled however as to how all this is put together. Do they use:
\begin{equation}
  G_c = \mathrm{LAI} \cdot \overline{Gs^i_g(z)} + \overline{Gns^i_g(z)}
\end{equation}
  }
    Thank you very much for your detailed account of the \emph{mosaic approach}. It is correct that we use Eq.~(\ref{eq:Gsto}) and Eq.~(\ref{eq:Gns}). As pointed out, we have not properly indicated the weighted mean in Eq.~(22) and Eq.~(29) in the manuscript. Additionally, there is also a typo in Eq.~(13) and Eq.~(22) ($G_\mathrm{sto}$ should have been $g_\mathrm{sto}$). Eq.~(22) should have read as follows:
    \begin{equation}
      \overline{g^\mathrm{m}_\text{sto}} = \sum_{n=0}^{N} f_{\text{L}, n}\cdot g^\text{m}_{\text{sto}, n}
    \end{equation}    
    All is put together in Eq.~(13):
    \begin{equation}
      \label{eq:Gsto_corr}
      G_c = \mathrm{LAI} \cdot \overline{g^\mathrm{m}_\mathrm{sto}} + \overline{G_\mathrm{ns}}
    \end{equation}
    We acknowledge that Eq.~(13) should not proceed the other equations and that this might have caused additional confusion. 
    
  \item {\color{blue} In any case, I think this approach has serious problems. Why average first for $G_s$ and then for $G_{ns}$ , when it is the fluxes ($F_k$ , or $V^i_{g,k}(z_\mathrm{ref}) \times \chi^i_\mathrm{avg}(z_\mathrm{ref})$) which need to be averaged? I also do not understand why they would use the same $u_*$ and $R_a$ for all land-covers. I think the authors need to make a case for their approach, or change it.}
    We have discussed the concerns raised within this comment and found that we have indeed made a mistake in our implementation of the \emph{mosaic approach} which forces us to revise our model and repeat the model experiments.
  \end{itemize}
  
\item {\color{blue} Calculation of $R_a$:}
  \begin{itemize}
  \item {\color{blue} $R_a$ in CTM3 appears to be calculated just once, and from a height of 8\,m. This means
that there is no consistency between the $R_a$ term and the underlying surface, which is
clearly wrong. The similarity equation for $R_a$ given in their Eqn~(2) is very standard and has been
used for decades (Garratt 1992). As pointed out by Ref~1, the equation as given has
errors.}
Eq.~(2) in our manuscript is indeed erroneous. The correct equation is (e.g., Erisman, 1994):
\begin{equation}
  R_a = \frac{1}{\kappa u_*}\left[{\ln{\left(\frac{z-d}{z_0}\right)}-\Psi_m\left(\frac{z-d}{L}\right)+\Psi_m\left(\frac{z_0}{L}\right)}\right].
  \label{eq:aero_res}
\end{equation}

\item {\color{blue} The correct equation will not give negative values unless presented with wrong
inputs, and I suspect that that is what has happened. It is actually difficult to tell what
was tested from the manuscript though, since they state simply that $d$ is ’typically 0.7\,m’.
Did they use $d$ properly, consistent with the underlying land-cover and its $z_0$? Did
they assume that their 8\,m meteorology was at a physical height of 8\,m, or at $d + 8\,\mathrm{m}$.
If the latter, which $d$?}
  At that point, it is indeed neither clear from the manuscript nor from our internal documentation. The assumption that Eq.~(2) (in the manuscript) would become negative, was most likely based on a wrong form of the integrated stability function $\Psi_m$.

  Given $z = z_\mathrm{ref} = 8\,\mathrm{m}$ and $d = 0.7\cdot 1\,\mathrm{m}$, it is right that neither the correct Eq.~(\ref{eq:aero_res}) nor the erroneous equation will result in negative $R_a$. The statement "$d$ is ’typically 0.7\,m’" is a typo. We correct: \emph{"[...] typically $d = 0.7\cdot h(N,\mathrm{lat})\,\mathrm{m}$ for vegetation other than forests [...]"}.

  Correction note: The average height of the lowermost model level is actually $20\,\mathrm{m}$ ($10\,\mathrm{m}$ for mid-level).
  
\item {\color{blue} Lines 19--30 of this section explain the Monteith alternative, but in a rather confusing
way. For example, when is $z_0$ ever zero, as stated on line 23, or why does $\partial_z R_a \rightarrow R_a $
for finite $z$? (I know what they intend to say, but it isn’t at all clear.)}
  The referee is right. The text and derivation of the Monteith relation is confusing and also erroneous. If we do not implement Eq.~(\ref{eq:aero_res}), we will elaborate on this in a revised version of the manuscript: \emph{"
\begin{equation}
  \Phi_{Q_\text{sensible}} = \rho_\text{air}\,c_P \cdot \frac{\Delta T}{R_{a, H}}.
  \label{eq:sens_heat}
\end{equation}
Herein, $\rho_\text{air}$ is the air density, $c_P$ the specific heat at constant pressure $P$, $R_{a, H}$ is the diffusion resistance to sensible heat between surface and the reference height $z$, and $\Delta T$ is the difference between the surface temperature $T_0$ and the temperature at a reference height $T_z$. From the eddy diffusion theorem, a similar formulation can be derived
\begin{equation}
  \Phi_{Q_\text{sensible}} = \rho_\text{air}\,c_P \cdot \frac{\partial T}{\partial u} \cdot u_*^2.
  \label{eq:eddy_theo}
\end{equation}
For finite $z$, $\partial T \rightarrow \Delta T$.
  "}

\item {\color{blue} In any case, here
the authors end up with a stability-independent equation for $R_a$, without mentioning or
discussing that fact.}
  In the revision of the model, we are going to implement the stability dependent $R_a$ following Eq.~(\ref{eq:aero_res}).
  
\item {\color{blue} This very shallow layer is also very problematic for deposition calculations in general,
since the model cell seems to be run here with horizontal dimensions of $2.25 \times 2.25^\circ$,
or about $250 \times 250\,\mathrm{km}$ near the equator, but a vertical mid-level (CTM3’s $z_\mathrm{ref}$) of
just 8\,m. Now, profiles of wind and depositing gases are very sensitive to the underlying
surface, and should be very different for forests or lakes for example. Any wind-speed
or friction velocity calculated from a model of such large horizontal resolution will necessarily give values
at 8\,m which reflect the whole grid. Deposition rates for a specific
land-cover will vary enormously depending on what else is in the grid-square. (Although not strictly comparable,
we showed in Schwede et al. (2018) that differences
between the grid-average and forest specific deposition rates of N-compounds could be
as much as a factor of two and up to more than a factor of five in extreme cases. These
differences were largely dependent on how much forest occupied each grid cell.)}
  Thank you for pointing this out. In S2012 we find, $R_a$ evaluated "at around $45\,\mathrm{m}$" which is similar to the center of our second lowest model level. We can easily correct for this in a revised model version. 
  \end{itemize}
  
\item {\color{blue} Why so much focus on sea areas?: The text seems rather unbalanced with regard to the different land-covers. Sect. 2 uses 1/2 page on various $z_0$ corrections for oceans, but say nothing about the ecosystem
where ozone is expected to deposit at high rates: forests, crops, and other terrestrial
ecosystems. The supplementary has three Figs related to this oceanic deposition. Why?}
  The referee is right that ozone is predominately deposited to terrestrial ecosystems, which cover about 2/9 of the Earth's surface, while 2/3 is covered by oceans. Nevertheless, due to the ocean's vastness, relatively small changes in modeled dry deposition easily accumulate and influence the atmospheric ozone concentration. Luhar et al. (2017, 2018) show that the current models may overestimate the oceanic ozone sink by a factor of three -- rendering the ocean a even smaller sink for ozone.
  
  This said, Section~2 as a whole consists of about 9 pages, of which 1/2 page is dedicated to oceans ($R_b$ computation) and 7 pages are dedicated to vegetation especially stomatal and non-stomatal conductance ($R_c$ computation). Hence, we do not see a significant unbalance in favor of oceans herein.

  Regarding the three figures with respect to ocean in the supplementary material: The reason for these is that we found some "legacy code" in the model (linear fit to dynamic viscosity of air $\mu(T)$) after we had finished the implementation of the new dry deposition scheme, run the experiments, and done the analysis on these. We had to verify that this had no significant influence on the results. 
  
\item {\color{blue} Use of the term 'EMEP scheme'?: Sect. 3 discusses the comparisons of $V_g$ in terms of ’EMEP scheme’ versus
  ’Wesely scheme’, and sensitivity tests are named e.g. ’EMEP\_offlight’. As noted above
the scheme implemented in CTM3 is very different to that implemented in the EMEP
model, so this is very misleading. Please rename your scheme to something else.
I am worried that readers might get the impression that it is the EMEP scheme which
is being tested here, but it certainly is not. [...]}
  We apologize for the misleading naming of the new dry deposition scheme in the Oslo~CTM3 which is (partly) based on the formulations in the publication the referee refers to as \emph{S2012}. By no means have we meant to offend any of the original authors of the EMEP/MSC-W model nor intended to misguide the readers. We will rename the revised scheme appropriately ($\rightarrow$\emph{mOSaic scheme}).
  
\item {\color{blue} Reproduction of material from S2012:}
  \begin{itemize}
  \item {\color{blue} As far as I can see, Table S1, S2 and S3 are taken directly from S2012, with no change to parameters. It is not usual to copy tables from the work of other authors in this way.
    Just refer to S2012 (and give Table number as help).}
    We will follow this advice and refer to the tables in S2012 accordingly.
    
  \item {\color{blue} Many of the equations are from S2012, and many not. I would like the authors to
    make this very explicit, so that readers are not confused as to what comes from EMEP,
    and what has changed for CTM3.}
    We are going to elaborate on this matter in a revised version and refer to our equations' sources more properly.
  \end{itemize}
\end{itemize}
\section{Minor points}
\begin{itemize}
\item {\color{blue}P1 L22. Isn’t $\mathrm{H_2O}$ the most important greenhouse gas? (Say anthropogenic
  GHG perhaps?)}
  Thank you for pointing this out. We follow the advice and write: \emph{"[...] ozone is a potent anthropogenic greenhouse gas [...]"}
\item {\color{blue}P2 L3. The Wilson ref only concerns Europe, and its focus on the 95th percentile can hide trends
  found at higher percentiles (e.g. Simpson et al. 2014). A better ref would be Fleming et al. (2018) or Mills et al. (2018a). By the way, the most recent calculation on food security (using flux approaches) is now Mills et al. (2018b).}
  
  Thank you for pointing out these publications. We take them into consideration, when revising the manuscript.
\item {\color{blue}P2 L11. What does in situ mean here? Ozone production can take place over
  days of transport.}
  \emph{In situ} typically means \emph{on site, locally}. In this context, we used it exactly in this meaning. We elaborate on this in a revised version of the paper: \emph{Elevated ozone levels at a site may originate from both, the local production of ozone from its precursors, which are transported, and from advection of ozone itself. Long-range ozone transport occurs and might be most important in regions that else lack precursors. Tropospheric ozone is produced in complex photochemical cycles involving precursor gases [...]}
  
\item {\color{blue}P2 L25--35. This text about halogens is not really relevant to a dry deposition
paper. Reactions with bromine can be important sinks, but are not usually counted
as deposition.}
  The removal of ozone from the Arctic boundary layer in the presence of halogens is indeed not part of ozone dry deposition and we remove this reference in this context.
Although these processes are not relevant with respect to ozone dry deposition, they matter regarding the observed ozone abundances in the Arctics. One of the proposed mechanisms to trigger bromine explosions involves, e.g., an "initialization" by ozone dry deposition.  
\item {\color{blue}P3 L2--3. Why specify mid-June maximum for ozone. Monks (2000) might
  take issue with that, as would for example Sinha et al. (2015).}
  The referee has got a point here. Of cause the maximum of ozone in the annual cycle depends on the location. We remove this specification.
\item {\color{blue}P3 L4--5. There are plenty of ozone measurements made outside Europe. The
authors appear to be unaware of the massive ozone collections made under the
TOAR project (see e.g. Flemming, Mills refs below), or the high quality data
available from GAW (inha 2015).}
  Thank you for reminding us of the data collected under the TOAR project. We knew about this data set but have to admit that we have not made much use of it, yet. We have, however, not stated that there are, in general, no sites outside of Europe, but that long-term observations (started before the 1950s) are only found in Europe. Maybe the term "long-term" is not clear enough. We add: \emph{"From the observational side, the number of long-term observations (started before the 1950s) is limited and restricted to mainly European sides."}
  
\item {\color{blue}P3 L16. One also has dry deposition to water, as this paper makes clear later on.}
  That is indeed true, but we didn't want to name all possible surfaces onto which dry deposition takes place. We, hence, shall write: \emph{"Removal of any substance from the atmosphere which is not involving rain, e.g., through gravitational settling or by uptake by plants and soil, is referred to as dry deposition.}
  
\item {\color{blue}P3 L20. One usually refers to dry deposition as something between a near-
surface height (e.g. z = 1m, 10m, or 50m) and the surface, not from $z_0$. In fact,
at $z = z_0$ one has $u_z = 0$, and hence the author’s $R_a$ just below should be zero.}
Thank you for pointing out this inaccuracy. We, of cause, meant "near-surface" height or lowermost model level (which is indeed not at $z_0=0$). We change the text accordingly: \emph{"[...] dry deposition is a product between near-surface ozone concentration $\mathrm{[O_3](z_0)}$ (e.g. the lowermost model level) [...] "}
  
\item {\color{blue}P4 L20. I would remove the term textbook knowledge, since there are many
different approaches to nearly all these equations. It is thus good that the equations
as used in CTM3 are spelled out explicitly.}
  We follow the advice and remove the sentence in which the term occurs and write: \emph{"[...] we will give a detailed account of the new dry deposition scheme and the equations that we use."}
  
\item {\color{blue}P5 L5. I would add Emberson et al. (2000a) and Tuovinen et al. (2004) to
the list of EMEP refs here, since this was the first publication of the methods that
have more or less been used until today.}
  We have added the references regarding the EMEP MSC-W model. \emph{"[...] we mainly follow Simpson et al. (2012) in their description of dry deposition used in the European Monitoring and Evaluation Programme (EMEP) MSC-W model (see also, Emberson et al., 2000; Simpson et al., 2003; Tuovinen et al., 2004;), [...]"}
  
\item {\color{blue}P7, notation. In S2012 and EMEP generally, we use upper-case G and R to refer
to canopy-scale (bulk) variables, and lower-case for leaf-scale. Thus, in EMEP
we would have $G_\mathrm{sto} = LAI g_\mathrm{sto}$. Here the authors seem to mix upper and lower
case between their equations (13) and (14).}
  We have indeed mixed up the cases here.
  
\item {\color{blue}P7 Eqn (13). Is LAI one-sided, 2-sided, projected .... define.}
  LAI is one-sided taken from ISLSCP2 FASIR. We add this to the sentence: \emph{"[...] $\text{LAI}$ is the one-sided leaf area index taken from ISLSCP2 FASIR [...]"}
  
\item {\color{blue}P8 L1. S2012 do not suggest using depths lower than 1\,m. We use SMI3 which
  is from 28-100 cm.}
Thank you for the clarification. This may have been a misunderstanding on our side. Since we have to repeat our simulations, we will choose the appropriate SWVL from the start.
  
\item {\color{blue}P8 L2. Why did you choose to use the surface (0--7\,cm) soil moisture?}
  Initially we choose 0--7\,cm due to the original scope of this work within our research project, in which we study ozone concentrations and uptake by plants in boreal regions where the soil columns are expected to be more shallow then, e.g., at mid-latitudes. We expect the stomatal conductance in these environments to be more sensitive to precipitation. Using the actual water availability to vegetation, would be better, but this is not feasible without a proper land surface model. 
  
\item {\color{blue}P8 L18. This is wrong. Nothing in the EMEP model is used to ’mimic the
time lag..’. We use the light function to modify stomatal conductance, as with the
other $f$ factors.}
  We change this inadequate formulation: \emph{"[...] this integrated photon flux is used to modify the stomatal conductance in response to light."}
  
\item {\color{blue}P9 L20, and Table 1. The consequence of Table 1 is that vegetation at $0.5^\circ\,\mathrm{N}$
will start growing at day 90, whereas those at $0.5^\circ\,\mathrm{S}$ will start on day 272. (By the
way, in EMEP now we use monthly factors from the LPJ-GUESS model to derive
phenology for non-European areas, because of such difficulties with tabulations.)}
  Thank you for the remark. We see this problem. Using an actual land model would solve this, but at that point it is not feasible to integrate such in the Oslo~CTM3.
  
\item {\color{blue}P9 L18. What do you mean by "surface or 2\,m"? Surface might refer to skin or leaf temperature?}
  This may have been indeed unclear. It is the $\mathrm{2\,m}$ temperature. We change the text accordingly: \emph{"[...] $\theta_0$ is the $2\,\mathrm{m}$ temperature in $\mathrm{^\circ C}$."}
  
\item {\color{blue}P9 Eqn (26). Say 1st and 2nd, not I. and II.}
  We follow the advise regarding Eq.~(26) and also change Table~1 accordingly.
  
\item {\color{blue}P9 Eqn (27). This equation is a modification of Erisman’s original (1994) formulation, so explain that.}
  We follow the advise and write: \emph{"The in-canopy resistance $R_\text{inc}$ (Erisman et al., 1994) is then modified with respect to each (vegetated) land type $N$ [...]"} {\bfseries ToDo -- To Amund: Were does this modification come from?}
  
\item {\color{blue}P10 L12. Be explicit that this statement refers to S2012. The current EMEP
  model uses different heights for e.g. tropical vegetation.}
  We follow this advice and write: \emph{"The vegetation height $h(N, \text{lat})$ as described by Simpson et al. (2012) [...]"}
  
\item {\color{blue}P11 Sect 2.2. I also found this aerosol section confusing. Eqn (30) is from S2012,
and so is the factor 0.008.SAI/10 used in Eqn (33), but here new $a_1$ coefficients
are defined. Did the ’aerosol microphysic model’ referred to also mix equations
in this way? Is there any publication as to the reliability of this method?}
  The $a_1$ coefficients were calculated from deposition velocities taken from the Oslo~CTM2 (\href{http://folk.uio.no/mariantl/osloctm3/history.html}{Oslo~CTM2 list of publications}). We meant to use this new scheme for aerosols in the Oslo~CTM3, but as of now it has not been evaluated and is actually not used. Hence, we will remove this section from the paper and add: \emph{"In addition to the gaseous species, Simpson at al. (2012) also modify aerosol deposition velocities, namely black carbon (BC) and organic carbon (OC), sulfuric aerosols ($\mathrm{SO_4}$, $\mathrm{MSA}$) and secondary organic aerosols (SOA), but we have not updated our model with respect to these."}.
  
\item {\color{blue}P12 Fig.2. I didn’t understand what is being done here. The Figure suggests that
the EMEP scheme has one category for ’Forests, Med. scrub’, whereas S2012
lists 4 types of forest, as well as Mediterranean scrub as a separate ecosystem.
This figure also suggests that EMEP has savanna, which it doesn’t, but do have
many other categories (Table 3 of S2012 lists 16 main categories. The current
EMEP model has 32.)}
  The caption may become clearer after we changed the name of our scheme, so that it can no longer be confused with the actual EMEP model. The purpose of the mapping is to project the more detailed categories for stomatal conductance to the 10 ground-surface resistance categories (Table~S19 in S2012 supplement).
  
\item {\color{blue}P12 L13. Again, the current EMEP model is not eurocentric, and uses global
  phenology calculations.}
  We follow the advice and drop the term "eurocentric" the text: \emph{"Since the parameterization of SGS and EGS in Simpson et al. (2012) is not applicable [...]"}
  
\item {\color{blue}P13 Sect 2.3.2. The initial lines (14-16) are hard to understand and only by
reading further do I see what they mean by ’de-accumulated’. If working with
IFS PPFD is so hard, why didn’t the authors just calculate hourly (or minute-by-
minute) PPFD using cloud-cover and zenith angles?}
We have not used that ansatz, because we do not have the radiation fields in the relevant wavelengths available from OpenIFS output.
\end{itemize}
\newpage

\end{document}
