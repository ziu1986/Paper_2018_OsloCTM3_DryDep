% Authors' response
\documentclass{scrartcl}
\usepackage[utf8]{inputenc}
\usepackage[english]{babel}
\usepackage{xcolor}
\usepackage{url}
\usepackage{multirow}
\thispagestyle{empty}

\begin{document}
\section*{Authors' response}
To gmd-2019-21 David Simpson (10 Apr 2019):
We appologize for the late response regarding the comments on our paper. We had to evaluate the major concerns raised in the review and found that we have to revise our model and repeat the model experiments to address all concerns. Nevertheless, we will respond to the questions in more detail in the following.

\section{Major points}
\begin{itemize}
\item {\color{blue} Mosaic formulation: [...] In any case, I think this approach has serious problems. Why average first for $G_s$ and then for $G_{ns}$ , when it is the fluxes ($F_k$ , or $V^i_{g,k}(z_\mathrm{ref}) \times \chi^i_\mathrm{avg}(z_\mathrm{ref})$) which need to be averaged? I also do not understand why they would use the same $u_*$ and $R_a$ for all land-covers. I think the authors need to make a case for their approach, or change it.}
  Thank you very much for your detailed account of the \emph{mosaic approach}. We have discussed the concerns raised within this comment and found that we may have indeed made a mistake in our implementation of the \emph{mosaic approach} which forces us to revise our model and repeat the model experiments.
\item {\color{blue} Calculation of $R_a$:}
  \begin{itemize}
  \item {\color{blue} $R_a$ in CTM3 appears to be calculated just once, and from a height of 8\,m. This means
that there is no consistency between the $R_a$ term and the underlying surface, which is
clearly wrong. The similarity equation for $R_a$ given in their Eqn~(2) is very standard and has been
used for decades (Garratt 1992). As pointed out by Ref~1, the equation as given has
errors. The correct equation will not give negative values unless presented with wrong
inputs, and I suspect that that is what has happened. It is actually difficult to tell what
was tested from the manuscript though, since they state simply that $d$ is ’typically 0.7\,m’.
Did they use $d$ properly, consistent with the underlying land-cover and its $z_0$ ? Did
they assume that their 8\,m meteorology was at a physical height of 8\,m, or at $d + 8\,\mathrm{m}$.
If the latter, which $d$?}
    We have, in fact, not presented "wrong inputs" to the algorithm.
    As pointed out by Ref \#1, we have made a mistake when deriving the formulation for $R_a$, as it is presented in S2012, for the framework of the Oslo~CTM3. This sign-flip in the derived formulation let to negative values. Because we oversaw this flaw, we concluded that we would no be able to use the S2012 formulation of $R_a$ in our model and decided to use the Monteith alternative.
    Since we are going to revise the model, we will address this issue and fix the calculation of $R_a$. 

    \item {\color{blue} Lines 19--30 of this section explain the Monteith alternative, but in a rather confusing
way. For example, when is $z_0$ ever zero, as stated on line 23, or why does $\partial_z R_a \rightarrow R_a $
for finite $z$? (I know what they intend to say, but it isn’t at all clear.) In any case, here
the authors end up with a stability-independent equation for $R_a$, without mentioning or
discussing that fact. This very shallow layer is also very problematic for deposition calculations in general,
since the model cell seems to be run here with horizontal dimensions of $2.25 \times 2.25^\circ$ ,
or about $250 \times 250\,\mathrm{km}$ near the equator, but a vertical mid-level (CTM3’s $z_\mathrm{ref}$ ) of
just 8\,m. Now, profiles of wind and depositing gases are very sensitive to the underlying
surface, and should be very different for forests or lakes for example. Any wind-speed
or friction velocity calculated from a model of such large horizontal resolution will necessarily give values
at 8\,m which reflect the whole grid. Deposition rates for a specific
land-cover will vary enormously depending on what else is in the grid-square. (Although not strictly comparable,
we showed in Schwede et al. (2018) that differences
between the grid-average and forest specific deposition rates of N-compounds could be
as much as a factor of two and up to more than a factor of five in extreme cases. These
differences were largely dependent on how much forest occupied each grid cell.)}
\end{itemize}      
\item {\color{blue} Why so much focus on sea areas?: The text seems rather unbalanced with regard to the different land-covers. Sect. 2 uses 1/2 page on various $z_0$ corrections for oceans, but say nothing about the ecosystem
where ozone is expected to deposit at high rates: forests, crops, and other terrestrial
ecosystems. The supplementary has three Figs related to this oceanic deposition. Why?}
\item {\color{blue} Use of the term 'EMEP scheme'?: Sect. 3 discusses the comparisons of $V_g$ in terms of ’EMEP scheme’ versus
  ’Wesely scheme’, and sensitivity tests are named e.g. ’EMEP\_offlight’. As noted above
the scheme implemented in CTM3 is very different to that implemented in the EMEP
model, so this is very misleading. Please rename your scheme to something else.
I am worried that readers might get the impression that it is the EMEP scheme which
is being tested here, but it certainly is not. [...]}
  We appologize for the missleading naming of the new dry deposition scheme in the Oslo~CTM3 which is (partly) based on the formulations in the publication the referee refers to as \emph{S2012}. By no means have we meant to offend any of the original authors of the EMEP/MSC-W model nor intented to misguide the readers. We will rename the revised scheme apropriately. (\emph{mOSaic} maybe?) 
\item {\color{blue} Reproduction of material from S2012:}
  \begin{itemize}
  \item {\color{blue} As far as I can see, Table S1, S2 and S3 are taken directly from S2012, with no change to parameters. It is not usual to copy tables from the work of other authors in this way.
    Just refer to S2012 (and give Table number as help).}
    We will follow this advice and refer the tables in S2012 accordingly.
  \item {\color{blue} Many of the equations are from S2012, and many not. I would like the authors to
    make this very explicit, so that readers are not confused as to what comes from EMEP,
    and what has changed for CTM3.}
    We are going to ellaborate on this matter in a revised version and point out our equations' sources more clearly.
  \end{itemize}
\end{itemize}
\section{Minor points}
\begin{itemize}
\item {\color{blue}P1 L22. Isn’t H 2 O the most important greenhouse gas? (Say anthropogenic
GHG perhaps?)}
\item {\color{blue}P2 L3. The Wilson ref only concerns Europe, and its focus on the 95th percentile can hide trends
  found at higher percentiles (e.g. Simpson et al. 2014). A
better ref would be Fleming et al. (2018) or Mills et al. (2018a). By the way,
the most recent calculation on food security (using flux approaches) is now Mills
et al. (2018b).}
\item {\color{blue}P2 L11. What does in situ mean here? Ozone production can take place over
days of transport.}
\item {\color{blue}P2 L25--35. This text about halogens is not really relevant to a dry deposition
paper. Reactions with bromine can be important sinks, but are not usually counted
as deposition.}
\item {\color{blue}P3 L2--3. Why specify mid-June maximum for ozone. Monks (2000) might
take issue with that, as would for example Sinha et al. (2015).}
\item {\color{blue}P3 L4--5. There are plenty of ozone measurements made outside Europe. The
authors appear to be unaware of the massive ozone collections made under the
TOAR project (see e.g. Flemming, Mills refs below), or the high quality data
available from GAW (inha 2015).}
\item {\color{blue}P3 L16. One also has dry deposition to water, as this paper makes clear later
on.}
\item {\color{blue}P3 L20. One usually refers to dry deposition as something between a near-
surface height (e.g. z = 1m, 10m, or 50m) and the surface, not from $z_0$. In fact,
at $z = z_0$ one has $u_z = 0$, and hence the author’s $R_a$ just below should be zero.}
\item {\color{blue}P4 L20. I would remove the term textbook knowledge, since there are many
different approaches to nearly all these equations. It is thus good that the equations
as used in CTM3 are spelled out explicitly.}
\item {\color{blue}P5 L5. I would add Emberson et al. (2000a) and Tuovinen et al. (2004) to
the list of EMEP refs here, since this was the first publication of the methods that
have more or less been used until today.}
\item {\color{blue}P7, notation. In S2012 and EMEP generally, we use upper-case G and R to refer
to canopy-scale (bulk) variables, and lower-case for leaf-scale. Thus, in EMEP
we would have $G_\mathrm{sto} = LAI g_\mathrm{sto}$. Here the authors seem to mix upper and lower
case between their equations (13) and (14).}
\item {\color{blue}P7 Eqn (13). Is LAI one-sided, 2-sided, projected .... define.}
\item {\color{blue}P8 L1. S2012 do not suggest using depths lower than 1\,m. We use SMI3 which
is from 28-100 cm.}
\item {\color{blue}P8 L2 - why did you choose to use the surface (0--7\,cm) soil moisture?}
\item {\color{blue}P8 L18. This is wrong. Nothing in the EMEP model is used to ’mimic the
time lag..’. We use the light function to modify stomatal conductance, as with the
other $f$ factors.}
\item {\color{blue}P9 L20, and Table 1. The consequence of Table 1 is that vegetation at $0.5^\circ\,\mathrm{N}$
will start growing at day 90, whereas those at $0.5^\circ\,\mathrm{S}$ will start on day 272. (By the
way, in EMEP now we use monthly factors from the LPJ-GUESS model to derive
phenology for non-European areas, because of such difficulties with tabulations.)}
\item {\color{blue}P9 L18. what do you mean by "surface or 2\,m"? Surface might refer to skin or
leaf temperature?}
\item {\color{blue}P9 Eqn (26). Say 1st and 2nd, not I. and II.}
\item {\color{blue}P9, Eqn (27). This equation is a modification of Erisman’s original (1994) formulation, so explain that.}
\item {\color{blue}P10 L12. Be explicit that this statement refers to S2012. The current EMEP
model uses different heights for e.g. tropical vegetation.}
\item {\color{blue}P11 Sect 2.2. I also found this aerosol section confusing. Eqn (30) is from S2012,
and so is the factor 0.008.SAI/10 used in Eqn (33), but here new a 1 coefficients
are defined. Did the ’aerosol microphysic model’ referred to also mix equations
in this way? Is there any publication as to the reliability of this method?}
\item {\color{blue}P12 Fig.2. I didn’t understand what is being done here. The Figure suggests that
the EMEP scheme has one category for ’Forests, Med. scrub’, whereas S2012
lists 4 types of forest, as well as Mediterranean scrub as a separate ecosystem.
This figure also suggests that EMEP has savanna, which it doesn’t, but do have
many other categories (Table 3 of S2012 lists 16 main categories. The current
EMEP model has 32.)}
\item {\color{blue}P12 L13. Again, the current EMEP model is not eurocentric, and uses global
phenology calculations.}
\item {\color{blue}P13 Sect 2.3.2. The initial lines (14-16) are hard to understand and only by
reading further do I see what they mean by ’de-accumulated’. If working with
IFS PPFD is so hard, why didn’t the authors just calculate hourly (or minute-by-
minute) PPFD using cloud-cover and zenith angles?}

\end{itemize}
\newpage

\end{document}
