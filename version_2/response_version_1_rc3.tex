\section*{Authors' response}
To gmd-2019-21 Anonymous Referee \#1 (01 May 2019):
We thank the referee for his comments. Taking all comments from both referees
into consideration, we find it necessary to revise our model and redo the simulations.
\begin{itemize}
\item {\color{blue}  The authors state in their response that \emph{"In fact, we originally used Eq. (2) with L in
the denominator for the second term. The sign error was likely the reason why the
Monteith method was chosen. Certainly, an update shall be considered in the future,
but it is not feasible to redo all simulations now."}
The fact of the matter is that there have been some fundamental inconsistencies in
the formulation of this equation, and the authors’ logic of "it is not feasible to redo all
simulations now" is not tenable. Ideally, the simulation would need to be done again. At
the very least, the authors need to perform some test runs which demonstrate whether
or not the inconsistencies in Eq. (2) make any significant difference to the results.
The authors have also not clarified how the value of the zero-plane displacement height
d is selected. They simply state that d is constant (typically 0.7m). Looking up any
atmospheric boundary-layer text book (e.g. Garratt, 1992), d is approximately 0.7 times
the canopy height (or the height of the roughness element). Please check this for
consistency too.}
  We will address all matters for which we have to repeat the simulations, e.g.,
  calculation of $R_a$ , output of dry deposition velocities, etc., at the same time. We will
  also check the definition of the replacement height $d$ in our formulation. 
\end{itemize}
