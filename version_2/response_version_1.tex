% Authors' response
\documentclass{scrartcl}
\usepackage[utf8]{inputenc}
\usepackage[english]{babel}
\usepackage{xcolor}
\usepackage{url}
\usepackage{multirow}
\thispagestyle{empty}

\begin{document}
\section*{Authors' response}
To gmd-2019-21 Anonymous Referee \#1 (02 Apr 2019):
We thank the referee for his/her useful comments and we will take them into account in the revised version of this paper.
In the following, we will respond to the questions in detail.
\begin{itemize}
\item {\color{blue}  Abstract L15--17: \emph{“While high sensitivity to changes in dry deposition to
    vegetation is found in the tropics, the largest impact on global scales is associated to
    changes in dry deposition to the ocean and deserts.”} The authors do not provide details
  in the paper as to what has changed in the updated scheme for such an impact.}
  We elaborate on the source of the \emph{"largest impact on global scales"} in Section~3.2.2 (P18 L4--11). But we have to admit that it might not be clear to which resistance term the changes in the prescribed dry deposition velocities apply. 
  \begin{itemize}
  \item {\color{blue} Is it the surface resistance ($R_c$) value? Or the other two resistances ($R_a$ and $R_b$)? What are the typical values?}
    It would be indeed interesting to look at the resistance terms separately. However, they are not available in our output files.
    Technically, it would be possible to force the model output but such would involve redoing the experiments and run for at least a couple of (model) weeks. In our formulation, the surface resistance $R_c$ includes both stomatal and non-stomatal conductance. In case, of non-vegetated surfaces, such as \emph{desert}, \emph{ocean}, and \emph{snow/ice}, the non-stomatal conductance, $G_\mathrm{ns}^\mathrm{O_3}$, is dominant. For water, $R_b = 10\,\mathrm{s\,m^{-1}}$ in most cases. Thus, $R_\mathrm{gs}^\mathrm{O_3} \approx R_c^\mathrm{O_3} \propto (v_\mathrm{DD}^\mathrm{O_3})^{-1}$ (for $R_c$ values see Table~S2 in the manuscript Supplement~S.3).
  \item {\color{blue} What value of $R_c$ for water has been used and how it compares with the value used in the Wesely scheme?}
    We have given the prescribed dry deposition velocities in Section~3.2.2:
    \begin{itemize}
      \item For the \emph{Wesely scheme}
        $v_\mathrm{water}^\mathrm{O_3} = 0.07\,\mathrm{cm\,s^{-1}}$ and
        \item for the \emph{EMEP scheme}
          $v_\mathrm{water}^\mathrm{O_3} = 0.05\,\mathrm{cm\,s^{-1}}$.
    \end{itemize}
    Surface resistances for water are thus $R_c \approx 1429\,\mathrm{s\,m^{-1}}$ (\emph{Wesely scheme}) and $R_c = 2000\,\mathrm{s\,m^{-1}}$ (\emph{EMEP scheme}), respectively.
  \end{itemize}

\item {\color{blue}  P22 Table 3:}
  \begin{itemize}
  \item {\color{blue}  Why the deposition values for ocean + ice + land do to add up to the total
    values reported, for all simulations?}
    Are we right to assume that the referee is trying to state that the values for ocean, ice, and land do
    not add up to the total global ozone dry deposition? If that is the case, he is right, for we exclusively
    selected gridboxes associated to more then 98\% to the three surface types (ocean, ice, and land), while
    \emph{total} comprises all gridboxes. We admit that this is not clear.
    We shall repeat the computation of the values using the proper weighting and adapt the numbers (Table~\ref{tab:ozone_sinks}).
    We may also add that \emph{ice} in this analysis only
    refers to regions at high latitudes that are permanently covered by ice and snow and hence does not take
    sea ice and mountain glaciers into account. In the Oslo~CTM3, as written elsewhere in the manuscript, we
    compute dry deposition on ice and snow based on meteorological data. Therefore, the given values in Table~3
    comprise ozone dry deposition on sea ice in the case of \emph{ocean} and on snow covered land during winter
    in case of \emph{land}.
    \begin{table*}[t]
  \caption{Total ozone dry deposition for the respective model experiment in $\mathrm{Tg\,a^{-1}}$. The global ozone dry deposition has been weighted by ocean, ice and, land fraction in each gridbox, respectively. \emph{Ice} herein refers to regions at high latitudes that are permanently covered by ice and snow.}
  \begin{tabular}{lccccccccc|cccr}
    \hline
    \multirow{3}{*}{Experiment} & \multicolumn{3}{c}{Ocean} & \multicolumn{3}{c}{Ice} & \multicolumn{3}{c}{Land} & \multicolumn{3}{c}{Total} & $\Delta^\dagger$\\
    & NH & SH & Global & NH & SH & Global & NH & SH & Global & NH & SH & Global\\
    & \multicolumn{3}{c}{$(\mathrm{Tg\,a^{-1}})$} & \multicolumn{3}{c}{$(\mathrm{Tg\,a^{-1}})$} & \multicolumn{3}{c}{$(\mathrm{Tg\,a^{-1}})$} & \multicolumn{3}{c}{$(\mathrm{Tg\,a^{-1}})$} & $(\mathrm{\%})$\\
    \hline
    Wesely\_type & 159.7 & 147.8 & 307.5 & 3.6 & 6.3 & 9.9 & 446.6 & 193.7 & 640.3 & 612.4 & 347.9 & 960.2 & 46.8\\
    EMEP\_full & 107.6 & 105.2 & 212.8 & 2.5 & 5.5 & 8.0 & 296.6 & 135.0 & 431.6 & 408.6 & 245.6 & 654.2 & 0.0\\
    EMEP\_offLight & 110.0 & 105.8 & 215.8 & 2.5 & 5.4 & 7.9 & 337.1 & 156.0 & 493.0 & 451.3 & 267.2 & 718.6 & 9.8\\
    EMEP\_offPhen & 108.0 & 105.3 & 213.2 & 2.5 & 5.5 & 8.0 & 301.6 & 139.1 & 440.7 & 413.9 & 249.9 & 663.7 & 1.5\\
    EMEP\_SWVL4 & 109.2 & 107.0 & 216.2 & 2.6 & 5.5 & 8.1 & 303.6 & 138.2 & 441.8 & 417.2 & 250.7 & 667.9 & 2.1\\
    EMEP\_ppgs & 107.6 & 105.2 & 212.7 & 2.5 & 5.5 & 8.0 & 299.5 & 135.3 & 434.9 & 411.4 & 246.0 & 657.4 & 0.5\\
    EMEP\_ppgssh & 108.3 & 105.4 & 213.6 & 2.5 & 5.5 & 8.0 & 306.8 & 142.8 & 449.6 & 419.4 & 253.6 & 673.0 & 2.9\\
    EMEP\_ppgssh\_ice & 107.4 & 105.3 & 212.7 & 1.2 & 1.6 & 2.8 & 303.0 & 144.1 & 447.2 & 413.0 & 251.1 & 664.1 & 1.5\\
    EMEP\_ppgs\_2005 & 106.9 & 103.5 & 210.4 & 2.6 & 5.4 & 8.0 & 297.9 & 131.8 & 429.7 & 409.2 & 240.7 & 649.9 & -0.7\\
    \hline
  \end{tabular}
  {$^\dagger$: Difference of global annual total relative to \emph{EMEP\_full}.}% Table Footnotes
  \label{tab:ozone_sinks}
\end{table*}
   
  \item {\color{blue}  The new land-based deposition values are much lower than what has
    been reported in previous studies (e.g. Hardacre et al., 2015) and the authors largely
    attribute this to the changes in the updated scheme for the desert surface type. However,
    the paper does not provide any observational support to back this up. }
    \begin{itemize}
    \item {\color{blue}  Are there any relevant deposition measurements (velocity or flux) that can be used for this purpose?}
      The only paper we know of is a study by G\"{u}sten~et~al.~(1996) in which measurements of ozone concentrations
      and fluxes onto the Sahara desert are described and dry deposition velocities deduced. We can include a more
      thorough discussion of our results with respect to the observed fluxes by G\"{u}sten~et~al.~(1996) in the
      revision of our manuscript. 
    \item {\color{blue}  At least, some comparison with ozone measurements (or even $\mathrm{O_3}$ reanalyses)
      should be provided for this surface type (and perhaps others) to see if the model is
      heading in the right direction with the updated deposition scheme.}
      Thank you for the advise. We will look into this.
    \end{itemize}
  \item {\color{blue}  It will also be useful to report the global ozone burden from the various
    simulations.}
    Since Table~3 is already at maximum width with respect to the the page width, we shall show the global
    tropospheric ozone burden in a separate table (see the following Table~\ref{tab:trop_ozone_burden}). If we compare our results
    with Stevenson~et~al.~(2006)
    ($344\pm 39\,\mathrm{Tg}$) and the number given in IPPC AR5 (2013) ($337\pm 23\,\mathrm{Tg}$), we find
    that the ozone burden in the Oslo~CTM3 is higher then the model average from the start (\emph{Wesely scheme}).
    The implementation of the \emph{EMEP scheme}
    increases the tropospheric burden by roughly 8\,\% (compare \emph{Wesely type} and EMEP\_full).
    \begin{table*}[h]
        \caption{Annual mean tropospheric ozone burden for all experiments and $1 \sigma$ standard deviation.}
        \centering
        \begin{tabular}{lrcl}
          \hline
          Experiment & \multicolumn{3}{c}{Trop. $\mathrm{O_3}$}\\
          &  \multicolumn{3}{c}{(Tg)}\\
          \hline
          Wesely type & 361 & $\pm$ & 21\\
          EMEP\_full & 392 & $\pm$ & 28\\
          EMEP\_offLight & 388 & $\pm$ & 26\\
          EMEP\_offPhen & 392 & $\pm$ & 27\\
          EMEP\_SWVL4 & 402 & $\pm$ & 31\\
          EMEP\_ppgs & 392 & $\pm$ & 28\\
          EMEP\_ppgssh & 391 & $\pm$ & 27\\
          EMEP\_ppgssh\_ice & 403 & $\pm$ & 31\\
          EMEP\_ppgs\_2005 & 386 & $\pm$ & 26\\
          \hline
        \end{tabular}
        \label{tab:trop_ozone_burden}
    \end{table*}
  \end{itemize}
  
\item {\color{blue}  Section 2.1.1, Eq. (2): The statement \emph{“For certain values of z, z0, and L, this
    may result in nonphysical (negative) values for Ra.”} I do not comprehend as to why
  this would occur since this equation is simply based on the well-used Monin-Obukhov
  similarity theory (MOST) for the surface layer. This occurrence would also imply negative wind speeds.
  Actually Eq. (2) is incorrect: the term $\psi_m((z-d)/z_0)$ should be
  $\psi_m((z-d)/L)$, and the sign of the third term on the right-hand side should be positive
  (not negative). Given that $(z-d) > z_0$ (assuming the model is formulated correctly), Eq. (2) should always yield positive values.\\
  Eqs. (3--5): I am not sure why Monteith (1973) needs to be invoked here. Given that
  the term in the square brackets on right hand side of Eq. (2) is equal to $k\cdot u(z)/u^{*}$ as per
  MOST, substituting this into Eq. (2) results in Eq. (5).
  Define $z$, $z_0$ in Eq. (2). The parameter $d$ is the so-called displacement height, and is
  not a constant (depends on the surface type).}
  The reviewer is indeed correct that this equation is wrong. In fact, we originally
  used Eq.~(2) with $L$ in the denominator for the second term. The sign error was likely the reason why the \emph{Monteith method}
  was chosen. Certainly, an update shall be considered in the future, but it is not feasible to
  redo all simulations now. In the revised version of the manuscript, we change the text from \emph{"In Simpson et al.
  (2003,2012) it is described as [...] fall back to the [...]"} to \emph{"For technical reasons, we
have used the [...]"}.

\item {\color{blue}  P2 L25-33: The first reference to the Oslo~CTM3 in the body of the paper is
  made here as \emph{“...we have not implemented any parameterization of these processes
    in the Oslo CTM3 as of now.”} Some brief introductory text is required here (or better at
  the start of the paragraph) to introduce the model properly. Also, the text between lines
  25 -- 33 on what is not considered in the model is too detailed to be here, so shorten
  and move it to Section 2.}
  \begin{itemize}
  \item The Oslo CTM3 is properly introduced in Section~2. Hence, we move the sentence in L30-32 about \emph{polar boundary layer ozone depletion} to Section~2:
    \emph{"Although the ozone depleting events in the polar boundary layer (Section~1) are important to understand surface ozone abundance in Arctic regions in spring-time, no parameterization of these processes is implemented in the Oslo~CTM3 as of now."}
  \item Given that the influence of VSLS on tropospheric ozone is indirect (through depletion of ozone in the upper troposphere -- lower stratosphere and subsequent STE), this reference (L32--33) rather belongs to the discussion in Section~4 and will be moved:
    \emph{"In particular, the STE depends on the stratospheric ozone abundance which is, e.g. affected by very short-lived ozone depleting substances (VSLS) (Warwick et al., 2006; Ziska et al., 2013; Hossaini et al., 2016; Falk et al., 2017) and not taken into account in the Oslo~CTM3."}
  \end{itemize}

\item {\color{blue}  P3 L19/L28 and P21 L34: There is a newer ozone dry
  deposition study by Luhar et al. (2018, ACP, 18, 4329-4348) which, using global ozone
  reanalyses and a more realistic process-based oceanic deposition scheme, estimates
  the total global deposition at $722.8 \pm 87.3\,\mathrm{Tg O_3 yr^{-1}}$, which includes an oceanic
  component of $98.4 \pm 30.0\,\mathrm{Tg O_3 yr^{-1}}$. These figures should be cited for comparison.}
  Thank you for pointing this out. We were not aware of this study and will compare our results and refer
  to it at the given places and within our discussion.
  \emph{
    \begin{itemize}
    \item "A newer study by Luhar et al. (2018), however, indicates much lower amounts ($722.8 \pm 87.3\,\mathrm{Tg\,a^{-1}}$)."
    \item "Based on the global atmospheric composition reanalysis performed in the\\ECWMF project Monitoring Atmospheric Composition and Climate (MACC) and a more realistic process-based oceanic deposition scheme, Luhar et al. (2018) found that the ozone dry deposition to oceans amounts to $98.4 \pm 30.0\,\mathrm{Tg\,a^{-1}}$."
    \item "But also the results of  Luhar et al. (2017, 2018) yield a $(19-27)\,\mathrm{\%}$ lower ozone dry deposition than the models participating in the model intercomparison, with deposition to ocean ranging between $(12-21)\,\mathrm{\%}$ of the total annual ozone dry deposition."
    \end{itemize}
  }

\item {\color{blue}  P11 Section 2.2: Since the present paper is about ozone dry deposition, this section
  seems like a distraction and hence should be omitted.}
  The referee is right in his/her assessment. We therefore omit this section in the revised version of our manuscript.
  
\item {\color{blue}  P14 L15-18: Anthropogenic, biomass burning, and biogenic emissions are included in the model. How are other emissions such as soil $\mathrm{NO_x}$, wetland methane, and oceanic methane and CO specified?}
  Emissions from soil and wetlands are computed by MEGAN. Resultant $\mathrm{NO_x}$ emissions are upscaled to match Global Emissions InitiAtive (GEIA) inventory.
  For oceanic emissions of $\mathrm{CO}$, we use predefined global fields (POET, available through ACCENT/GEIA, \url{http://accent.aero.jussieu.fr/database_table_inventories.php}). $\mathrm{CH_4}$ for oceans is taken from surface data from NASA's HYMN project (\url{https://www.esrl.noaa.gov/gmd/ccgg/trends_ch4/}) given for the years 2000--2004. We shall include this information in the revised manuscript.
  
\item {\color{blue}  P15 L4: The statement \emph{“Accidentally, we have used emissions for the year
  2014 instead of 2005.”} It is not clear what the consequences on the results are of this?}
  In the introduction to Section~3.1, we wrote \emph{"For all model integrations, the meteorological reference year is 2005. This choice only affects the comparison with data and multi-model studies that either perform analysis on decadal averages or differing years."}. We shall elaborate on the discussion on the implications based on our model results (EMEP\_ppgs vs EMEP\_ppgs\_2005) in the revision of the manuscript. Though, the major consequence of this is that, for the majority of our model experiments, one can neither directly compare observations for the years 2005 nor for 2014 directly to the model results. Surface ozone observations, in fact, show a strong interannual veriability in ozone dry deposition and ozone concentrations at the sites, but studying these in detail may be well beyond the scope of this manuscript.
  \emph{}
  
\item {\color{blue}  Section 3.2.1: }
  \begin{itemize}
  \item {\color{blue}  Section 3.2.1: In the Fig 5 discussion, although snow and ice is discussed, there is
    no discussion on the oceanic differences between the present study and Hardacre et
    al. (2015). This is particularly important for the Southern latitudes.}
    \emph{}
    
  \item {\color{blue}  The Hardacre et al. (2015) simulations were for the year 2001,
    whereas the present study is mostly for the year 2015 emissions (see Table 1) driven
    by the year 2005 meteorology. In addition, the observational averages used in Fig. 8
    are based on multi-year data. The authors should discuss the implications of these
    differences about different years on the deposition results presented (e.g. uncertainty).}
    \emph{}
  \end{itemize}
  
\item {\color{blue}  P24 L3--4: \emph{“The annual amount of ozone dry deposition decreases by up
to 100\% changing from the old dry deposition scheme to the new one.”} Table 3 does
not support this, but this may be true for some surface types. So please qualify the
statement.}
  We have indeed not specified our statement in the mentioned sentence, while we had done so elsewhere in the manuscript (P15 L12). We complete our statement in the revised version of the manuscript:
  \emph{"[...] ozone dry deposition decreases by up to $100\,\%$ over all major desert areas [...]. At the same time, it increases over tropical forest.}
  
\item {\color{blue}  P24 L15: \emph{“Most of the decrease in ozone dry deposition in the Oslo CTM3
can be attributed to changes in dry deposition velocities over the ocean and deserts.”}
What are the dominant factors in these changes? For example, is it mostly the surface
resistance ($R_c$) term? For the ocean, it is likely to be $R_c$. For deserts, maybe $R_b$? Is it
possible to quantify these differences in the resistance terms?}
  We have already answered the question with respect to ocean (see first bulletpoint). In summary, since $R_b$ is quite small in most of the cases, the dominant factor for the ozone dry deposition onto ocean is the surface resistance $R_c$ which is tabulated in Table~S2. Regarding ozone dry deposition onto deserts, we use Eqs.~(7--8) to deduce
  \begin{equation}
    R_b^i = \frac{2}{\kappa u_*} \cdot \left(\frac{D_\mathrm{H_2O}}{D_i} \cdot \frac{\mathrm{Sc}_\mathrm{H_2O}}{\mathrm{Pr}}\right)^{\frac{2}{3}},
  \end{equation}
  with $\mathrm{Pr}=0.72$, $\kappa=0.4$, $\mathrm{Sc}_\mathrm{H_2O} = 0.6$, $D_\mathrm{H_2O}/D_i = 1.6$.
  We estimate $u_*$ from Eq. (16.67) in Seinfeld~and~Pandis (2006)
  \begin{equation}
    u_*(h) = \frac{\kappa\cdot \overline{u_x}(h)}{\mathrm{ln}(h/z_0)},
  \end{equation}
  with $h = 8\,\mathrm{m}$, $z_0^\mathrm{desert}\approx 10^{-3}\,\mathrm{m}$, and windspeeds not exceeding a gentle breeze ($1.8\,\mathrm{km\,h^{-1}} \leq \overline{u_x}(h) \leq 28\,\mathrm{km\,h^{-1}}$), we find $272\,\mathrm{s\,m^{-1}} \geq R_b \geq 17\,\mathrm{s\,m^{-1}}$. This is $1-2$ orders of magnitude smaller than $R_c = 2000\,\mathrm{s\,m^{-1}}$ and thus not negligible for low windspeeds. In summary, $R_c$ is dominant in our formulation of dry deposition of ozone to deserts (unless we have calm wind conditions).
  \emph{}
  
\item {\color{blue}  P24 L24: \emph{“2-layer gas exchange with ocean waters (Luhar et al., 2017).”}
As mentioned earlier, Luhar et al. (2018) has derived a more realistic process-based
deposition scheme for the ocean, but the results for deposition velocity do not seem to
be too different from those in Luhar et al. (2017).}
  We acknowledge Luhar et al. (2018) an update the sentence:
  \emph{“[...] 2-layer gas exchange with ocean waters (Luhar et al., 2017, 2018).”}
  
\item {\color{blue}  P25 L11--12: The comment \emph{“This is most likely reflecting the ongoing
industrialization process of countries in the southern hemisphere and the commitment
and implementation of air quality regulations of industrialized nations in the northern
hemisphere”} is quite speculative and may be omitted.}
  We follow the kind advise of the referee and remove the sentence in the revised version of the manuscript.
  
\item {\color{blue}  Eq. (13) cf. Eq. (14): $g_\mathrm{STO}$ or $G_\mathrm{sto}$ -- use consistency with notation.}
  Thanks for pointing this out. We will change this in the revised version of the manuscript.
  
\item {\color{blue}  The first half of the abstract, the text before \emph{“In this paper...,”} is introductory
  material and can be deleted.}
  This is indeed the case and we will remove it in the revised version.
  
\item {\color{blue}  Abstract L15--16: it is better to say “...leading to an increase in surface
  ozone of up to 100\% in some regions.”}
  We follow the advice of the referee and change the sentence accordingly.
  
\item {\color{blue}  P22 L7: \emph{“At about 4 of 6 sites.” About? Not sure?}}
  Thanks for pointing out the misplaced \emph{"about"} in this sentence. We are certain regarding that number. 
  
\end{itemize}
\newpage

\end{document}
